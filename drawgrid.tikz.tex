%% Node labelling functions

% row, col, rowoffset, coloffset
\newcommand{\plainlabelnode}[4]{
\pgfmathtruncatemacro{\rowlabel}{#1+#3}\pgfmathtruncatemacro{\collabel}{#2+#4}$n_{\rowlabel,\collabel}$}

% row, col, rowoffset, coloffset, column-variable
\newcommand{\varcollabelnode}[5]{
\pgfmathtruncatemacro{\row}{#1+#3}\pgfmathtruncatemacro{\col}{#2+#4}\pgfmathtruncatemacro{\colzero}{\col-1}\newcommand{\collabel}{\ifnumcomp{\colzero}{=}{0}{#5}{\ifnumcomp{\colzero}{>}{0}{#5+\colzero}{#5\colzero}}}$n_{\row,\collabel}$}

% row, col, rowoffset, coloffset, row-variable
\newcommand{\varrowlabelnode}[5]{\pgfmathtruncatemacro{\row}{#1+#3}\pgfmathtruncatemacro{\rowzero}{\row-1}\pgfmathtruncatemacro{\col}{#2+#4}\newcommand{\rowlabel}{\ifnumcomp{\rowzero}{=}{0}{#5}{\ifnumcomp{\rowzero}{>}{0}{#5+\rowzero}{#5\rowzero}}}$n_{\rowlabel,\col}$}


% row, col, rowoffset, coloffset, row-variable, col-variable
\newcommand{\varlabelnode}[6]{\pgfmathtruncatemacro{\row}{#1+#3}\pgfmathtruncatemacro{\rowzero}{\row-1}\pgfmathtruncatemacro{\col}{#2+#4}\pgfmathtruncatemacro{\colzero}{\col-1}\newcommand{\rowlabel}{\ifnumcomp{\rowzero}{=}{0}{#5}{\ifnumcomp{\rowzero}{>}{0}{#5+\rowzero}{#5\rowzero}}}\newcommand{\collabel}{\ifnumcomp{\colzero}{=}{0}{#6}{\ifnumcomp{\colzero}{>}{0}{#6+\colzero}{#6\colzero}}}$n_{\rowlabel,\collabel}$}



\pgfkeys{/tikz/.cd,% to set the path
  rows/.get=\krows,
  rows/.store in=\krows,
  cols/.get=\kcols,
  cols/.store in=\kcols,
  rowoffset/.initial=0,
  rowoffset/.get=\krowoffset,
  rowoffset/.store in=\krowoffset,
  coloffset/.initial=0,
  coloffset/.get=\kcoloffset,
  coloffset/.store in=\kcoloffset,
  labeler/.get=\klabeler,
  labeler/.store in=\klabeler,
  labelerA/.get=\klabelerA,
  labelerA/.store in=\klabelerA,
  labelerB/.get=\klabelerB,
  labelerB/.store in=\klabelerB,
  labelerC/.get=\klabelerC,
  labelerC/.store in=\klabelerC,
  labelerD/.get=\klabelerD,
  labelerD/.store in=\klabelerD,
  northborder/.initial=outside,
  northborder/.get=\knorthborder,
  northborder/.store in=\knorthborder,
  southborder/.initial=outside,
  southborder/.get=\ksouthborder,
  southborder/.store in=\ksouthborder,
  eastborder/.initial=outside,
  eastborder/.get=\keastborder,
  eastborder/.store in=\keastborder,
  westborder/.initial=outside,
  westborder/.get=\kwestborder,
  westborder/.store in=\kwestborder,
}

%% Draws a grid - num rows, num cols, row offset, col offset
\newcommand{\drawgrid}[1]{{
     \tikzset{#1}


	\newcommand{\maxrows}{\krows}
	\newcommand{\maxcols}{\kcols}
	\newcommand{\rowoffset}{\krowoffset}
	\newcommand{\coloffset}{\kcoloffset}
	% Argument is a function: r,c -> node label
	\newcommand{\labeler}{\klabeler}

	\foreach \row in {1,...,\maxrows} {
		\pgfmathsetmacro{\ypos}{-2 * \row + \rowoffset}
		\pgfmathtruncatemacro{\prevrow}{\row - 1}

		\foreach \col in {1,...,\maxcols} {
			\pgfmathsetmacro{\xpos}{2 * \col + \coloffset}
			\pgfmathtruncatemacro{\prevcol}{\col - 1}
			\newcommand{\thisnode}{(n \row \space \col)}

			% Get node label
			\ifstrempty{\klabelerA}{
				\newcommand{\nodelabel}{\labeler{\row}{\col}}
			}
			\ifstrempty{\klabelerB} {
				\newcommand{\nodelabel}{\labeler{\row}{\col}{\klabelerA}}
			}
			\ifstrempty{\klabelerC} {
				\newcommand{\nodelabel}{\labeler{\row}{\col}{\klabelerA}{\klabelerB}}
			}
			\ifstrempty{\klabelerD} {
				\newcommand{\nodelabel}{\labeler{\row}{\col}{\klabelerA}{\klabelerB}{\klabelerC}}
			}
			{
				\newcommand{\nodelabel}{\labeler{\row}{\col}{\klabelerA}{\klabelerB}{\klabelerC}{\klabelerD}}
			}
			% Create node
			\node (n \row \space \col) at (\xpos,\ypos) {\nodelabel};

			% Line from node to the previous horizontal node
			\ifnumcomp{\col}{>}{1} {
				\draw (n \row \space \prevcol) -- \thisnode;
			}

			% Line from node to the previous vertical node
			\ifnumcomp{\row}{>}{1} {
				\draw (n \prevrow \space \col) -- \thisnode;
			}


			% West border lines
			\ifnumcomp{\col}{=}{1} {
				\draw[\kwestborder] \thisnode -- (\xpos-1, \ypos);
			}
			% East border lines
			\ifnumcomp{\col}{=}{\maxcols} {
				\draw[\keastborder] (\xpos+1, \ypos) -- \thisnode;
			}

			% North border lines
			\ifnumcomp{\row}{=}{1} {
				\draw[\knorthborder] \thisnode -- (\xpos, \ypos+1);
			}
			% South border lines
			\ifnumcomp{\row}{=}{\maxrows} {
				\draw[\ksouthborder] (\xpos, \ypos-1) -- \thisnode;
			}
		}
	}
}}