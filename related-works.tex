We present a selection of prior works which touch upon the area of mesh structure, or are related to it in appealing ways.
% \section{Structure is beneficial}



% - How do quad meshes with subregions of structure come about
% - Are they seen often


\section{Structure detection and extraction}


Of all the works discussed in this chapter, \cite{tautges2004moab}~by Tautges is the most similar. The paper discusses the merits of structured meshes in great depth. It also defines the concept of disparate structured regions in a mesh, and discusses strategies for fusing them into a hierarchy of structures. A structure construction algorithm is also discussed very briefly.


% Fast Neighborhood Search on Polygonal Meshes
% http://www.inf.ethz.ch/personal/dpanozzo/papers/EGIT11-RocDeGPanPup.pdf
% Dude et al tackle the specific problem of neighborhood search by defining a spatial index, a hierarchy of clusters of vertices. Their method yields an approximation, ours is exact, with the mesh model left completely unaltered. However, we can learn a few things.

% The problem of indirections and its derivatives (non-contiguous memory access and its impact on cache hierarchy) is cited/highlighted as a key motivation in basing their mesh data structure on a geometric model, i.e. the spatial index. We undertake the opposite approach, viewing geometric data as mere associative data attached to a topological mesh model.

Adapting the mesh model to use a structured representation for the benefit of the relevant application domain is common practice.
Rocca et al.~\cite{rocca2011fast} address the particular problem of neighbourhood search: finding the set of vertices within a certain distance from a given vertex. They define a spatial index, a hierarchy of vertex clusters, that yields good approximations to neighbourhood search queries.

The problem of memory access indirections, associated with representing the mesh topology, is cited as a key motivation for basing their mesh data structure (the aforementioned spatial index) on a geometric model which does not require storing topological information. This is in sharp contrast to our approach of transforming the mesh topology representation, and treating the mesh geometry as arbitrary associated data.



% Motorcycle Graphs: Canonical Quad Mesh Partitioning
% http://www.ics.uci.edu/~goodrich/pubs/geomproc.pdf
% Whilst the motivations are different (mesh compression AND mesh isomorphism) the approach is similar:
% ``The simplest quadrilateral meshes are structured meshes, in which connections between quadrilaterals form a regular grid, but in complicated domains it may be necessary to use semi-regular meshes in which this structure is interrupted by a small number of extraordinary vertices that do not have degree four. We study how to partition a semi-regular mesh into a small number of structured submeshes.''


% ``To ensure that isomorphisms will always be found when they exist, our partition must be canonical: the same mesh must always generate the same partition. The partitioning algorithm we describe has this property.''
% We do not care about this property per se. There may be multiple optimal solutions for structure detection, for example.


% ``Additionally, these techniques may apply to areas beyond graphics such as scientific computation. Code for the finite element method can be greatly streamlined when applied to structured quadrilateral meshes. By partitioning unstructured meshes into structured submeshes, it should be possible to achieve similar speedups for semi-regular meshes.''

Eppstein et al.~\cite{eppstein2008motorcycle} describe a method for partitioning quadrilateral meshes into structured regions using the \emph{motorcycle graph} construction, whose inspiration came from the 1982 movie Tron. The algorithm works by placing particles on each \emph{extraordinary} vertex\footnote{\label{footnote:extraordinary-vertices} This is in contrast to an \emph{ordinary} vertex, which the authors define as ``a non-boundary vertex incident with four edges or a boundary vertex incident with at most three edges.''} in the mesh, and advancing them along edges until they collide. The enclosed regions formed by the paths represent structured regions.

Whilst the motivations are different, with the emphasis on applications in mesh compression and detecting mesh isomorphisms, the approach may be applicable to the scientific computation domain. Indeed, the authors state this explicitly:
\begin{quote}
These techniques may apply to areas beyond graphics such as scientific computation. Code for the finite element method can be greatly streamlined when applied to structured quadrilateral meshes. By partitioning unstructured meshes into structured sub-meshes, it should be possible to achieve similar speedups for semi-regular meshes.
\end{quote}

We note, however, that the method makes a stronger assumption about the mesh, in that its ``structure is interrupted by a \emph{small number} of extraordinary vertices that do not have degree four [emphasis added]''.


% There exist processes which explicitly attempt to produce semi-structured meshes, for example *BELOW* which uses domain specific knowledge about the source object to find long strips which can be made into quads.
% Automatic Decomposition and Efficient Semi-Structured Meshing of Complex Solids
% http://www.imr.sandia.gov/papers/imr20/Makem.pdf

Makem et al.~\cite{makem2012automatic} utilise properties of the modelled object to generate meshes with structured regions inherent. The presented method detects long thin shapes with simply defined geometry, such as length and curvature, and generates an appropriate structured region representing these shapes.



\section{Structure in parallel computation}
There is a plethora of work on mesh partitioning optimised for parallel computation. The techniques presented often frame their objectives around these maxims:
\begin{enumerate}
\item \label{item:maximize-locality} Maximize intra-partition locality, thereby minimizing cross-partition communication.
\item \label{item:minimize-size} Minimize partition size, subject to it being sufficiently large to counterbalance the communication overheads.
\item \label{item:maximize-number} The partitions should be balanced and plentiful in number, so as to utilise parallelism.
``A parallel computation is often only as efficient as the evenness with which its workload is distributed over the processors in a parallel machine.''
\end{enumerate}

Objective~\ref{item:maximize-locality} is certainly a desirable property for our structured regions, and is in fact precisely what we set forth to perform.
On the other hand, objectives~\ref{item:minimize-size} and \ref{item:maximize-number} are rather misaligned with our needs; indeed, a monolithic structured region spanning the entire mesh would represent the best case for us.

Thus some of the techniques which hold these conflicting\footnote{The objectives are in conflict in our context. These objectives are more suited for their parallel context, naturally} objectives may not be fully compatible, though they undoubtedly offer helpful inspiration.

With that said, extending our techniques towards parallel computation is an obvious future step, and such future works should likely re-evaluate their objectives in lieu of this.



% Guide to Partitioning Unstructured Meshes for Parallel Computing
% http://www.hector.ac.uk/cse/reports/unstructured_partitioning.pdf
% Very briefly describes several tools used for partitioning the mesh

A technical report by Ridley~\cite{ridley2010guide} briefly discusses methods for partitioning a mesh so as to maximize spatial data locality, in other words adjacent elements tend to have memory locations that are close. The partitioning is applied by recursively bisecting the mesh geometrically, such that geometrically close points tend to cluster together. The emphasis of the report is towards improving the performance of parallel computation, but the benefits of partitioning extend to serial processing as well.



% Hierarchical Partitioning Techniques for Structured Adaptive Mesh Refinement Applications
% http://www.s3lab.ece.ufl.edu/publication/jsuper.pdf
% Partitions mesh to exploit parallel computation whilst minimizing communication costs. This is done by \emph{exploiting} the hierarchical structure in a mesh. The approach focuses on partitioning for purposes of parallel computation, which involves a) minimizing communication costs; and b) maximizing locality within partitions. We focus solely on B.
% For **future works**, however, A would come into the picture as well, which would add an interesting dimension to the problem for further exploration.
% ``This scheme uses space filling curves (SFC), which are a class of locality preserving recursive mappings from n-dimensional to 1-dimensional space.''
% ALSO, note that this is *dynamic*. We can do that too in the future? :)

Li et al.~\cite{li2004hierarchical} follow a similar theme of parallel computation, although their methods deal with adaptive mesh refinement, where a mesh is dynamically refined in regions with a high calculation error. The refinement process induces a hierarchical structure, which the presented partitioning algorithms aim to exploit.



% Hierarchical hybrid grids: data structures and core algorithms for multi-grid
% http://onlinelibrary.wiley.com/store/10.1002/nla.382/asset/382_ftp.pdf?v=1&t=hw5h84yv&s=bbe650e186f348823036d7426c24bd554ea00c1b

Bergen et al.~\cite{bergen2004hierarchical} employ structure-aware mesh refinement techniques, which construct a hierarchy of structured regions with each iterative refinement to the mesh. It is suggested that different element types, such as edges and faces, refined and stored separately, such that their distinct structuredness can be represented.

Since we use a simple structure representation, we make use of augmented structured regions, with different elements' structured region represented in a hierarchy.



% Stencils and Problem Partitionings: Their Influence on the Performance of Multiple Processor Systems
% http://ieeexplore.ieee.org/ielx5/12/35266/01676980.pdf?tp=&arnumber=1676980&isnumber=35266
% Partitioning a mesh based on its memory-access stencil. Again this is parallel oriented. Could be useful in defining the shape of the map access!

Reed et al.~\cite{reed1987stencils} focus on obtaining partitions best-suited to the \emph{stencil structure} associated with a particular computation, that is its neighbour-access pattern. They derive partition shapes for some common stencil structures, optimised to minimise inter-partition communication costs.

Tang et al.~\cite{tang2011pochoir} present a full-fledged \emph{Pochoir Stencil Compiler}. It specifies a domain-specific language that allows users to write a higher-level specification of a stencil computation embedded in C++ code. The compiler then automatically generates very efficient \emph{cache-oblivious}\footnote{The authors of~\cite{frigo1999cache} define a cache oblivious algorithm as that which ``[does not contain] parameters (set at either compile-time or runtime that can be tuned to optimize the cache complexity for the particular cache size and line length''} parallel loops that execute the stencil computation.

As discussed in subsection~\ref{subsec:given-kernel-function}, this is similar to our approach, where the structured regions are detected on the basis of relation-maps, such as cell-vertex or edge-cell maps.


% \url{https://www.cs.sfu.ca/~bgb2/personal/papers/nand11miccai.pdf}
% Bruv et al describe in [REF] the ``[construction of] curvature based features detectors to detect tube-like and sheet-like structures in DTI [diffusion tensor MRI]''. Differential equation-based methods are applied to characterize the ``'structured-ness'' of various components of the generated image, in terms of feature detection. Our focus is on detecting structuring in a precise manner, rather than a characteristic approach.

% FOR FUTURE WORKS: The differential equation approach may be an interesting tangent for future work, for instance applying it to geometric information to deduce areas of likely structure.


% \url{http://ieeexplore.ieee.org/ielx7/90370/06476010.pdf?tp=&arnumber=6476010&isnumber=6490370}
% Again, approximate/heuristical image-based structure detection. The paper mentions low-level and high-level pattern detection.




% multiblock structured mesh partitioning algorithm
% http://www.geuz.org/pipermail/getdp/2000/000138.html
% Michael sends an email discussing stuff







% Structured Grid Computational Pattern
% (EITHER FROM:
%   Course: CS 4800, Fall 2010, School: Northeastern
%   OR
% Parallel Computing Laboratory - Berkley University)
% http://parlab.eecs.berkeley.edu/wiki/_media/patterns/structuredgrid-2.pdf
% http://view.eecs.berkeley.edu/wiki/Structured_Grids
% Good overview of the benefits of structured regions for computation.



\section{Exploiting structure}

% Approximate Topological Matching of Quadrilateral Meshes
% http://www.ics.uci.edu/~goodrich/pubs/approx-match.pdf
% Find matching subgraphs of two meshes, that is having the same topology.

% ``Our reason for using mesh compression instead is based on the desire to speed up the computation time in an algorithm that operates on quad meshes. Thus, our approach actually fits the spirit of other algorithms (e.g., see (33; 15; 14; 38)) that perform data compression so as to improve algorithmic performance.''
% Mentions reducing mesh representation for the purpose of improving computation time.


Eppstein et al.~\cite{eppstein2008approximate} explore the problem of approximate topological matching between given quadrilateral meshes, that is detecting isomorphisms between their submeshes. To this end they discuss various techniques based on \emph{particle shooting}\footnote{The motorcycle graph construction discussed in the paper by the same authors~\cite{eppstein2008motorcycle} is based on particle shooting.}: ``particles'' are placed on certain vertices (for example, extraordinary vertices\footref{footnote:extraordinary-vertices}) and are then ``'fired'' along the mesh using certain rules.

One such algorithm, termed ``The Greedy Algorithm'' by the authors, is shown to suffer from a problem when particles advancing along the same front can get out of sync. The problem is remarkably similar to our discussion about contiguous detection in subsection~\ref{subsec:contiguous-detection}.


% Opportunistic Data Structures with Applications
% http://people.unipmn.it/manzini/papers/focs00draft.pdf
% Discusses implicit data structures which minimize storage overheads due to auxiliary information


% \section{}


% Predictable memory access patterns versus *known* memory access pattern.



\section{Chapter summary}
In this chapter we discussed the similarities and differences between our work and that of a variety of published works centred at or homing around our scope of interest. We saw that the inefficiencies associated with unstructured representation are a common pain point, and we saw the great interest in the area of detecting structure to aid parallel computations. Some works presented exploited structure for other intriguing purposes.
