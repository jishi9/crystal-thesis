% \newcommand{\imagewidth}{0.8\textwidth}

\section{Structured region boundaries}
%% TODO REFERENCE generalized definition
When running a core-computation over structured data, we would like to ensure that direct neighbour accesses are always within bounds. To this end, we define the concepts of \emph{interior} structured elements and \emph{fringe} structured elements, with respect to a given relation-map.

Let $R: S \mapsto S$ be a relation-map on an element set $S$. We assume that $S$ has been partitioned into $k$ structured regions $S_1$ to $S_k$, as well as the remaining unstructured partition $S_0$.
Let $e \in S_i$ be some structured mesh element in some structured region $S_i$.

We say that $e$ is an \emph{interior structured element} iff all neighbours $n$ of $e$, as defined by $R$, are structured mesh elements in $S_i$. Stated formally:
$$\forall n \in S. R(e,n) \implies n \in S_i$$
Conversely, a \emph{fringe structured element} is one which does not satisfy this property, that is it has some neighbour in $R$ which is not found within the structured region $S_i$.

Figure~\ref{fig:fringe-cells} exemplifies this for structured cell regions.

\begin{figure}
\includesvg[width=\imagewidth, svgpath=images/renumbering/]{fringe-cells}
\caption{The image depicts the boundaries induced by a cell-cell relation-map. The two structured regions are coloured in blue and red. Lighter shades denote interior structured cells, and darker shades denote fringe structured cells.}
\label{fig:fringe-cells}
\end{figure}


\section{Laying out vertices}
\label{subsec:vertex-associative-data}
With the regions of structured vertices extracted, we would like to lay out the associative vertex data in a convenient order such that the neighbours' data may be accessed directly. Recalling our decision in~\ref{sentence:2d-array}, we choose to detect rectangular structured regions so as to place them in two-dimensional arrays.

% TODO REFERENCE FUTURE WORKS
This still leaves us to determine the data layout within a two-dimensional array. We use a row-major order as it enables very simple (cheap) neighbour address calculations; column-major order exhibits the same property and may been used equally. Alternative orderings, as well as the general description of space filling curves are displayed by bruv), and may be worth exploring in future works.

Then given a rectangular structured vertex region $\Structured$ with $m$ rows and $n$ columns, we can represent its associative data in a two-dimensional array $Dat$ in row-major order. Clearly, the associative data of \emph{any} structured vertex $n_{r,c} \in \Structured$ can be accessed by $Dat[r][c]$. Similarly, we can access the associative data of the vertex neighbours of any \emph{interior} structured vertex $n_{r,c} \in \Structured$ by $Dat[r][c+1]$, $Dat[r][c-1]$, $Dat[r+1][c]$, and $Dat[r-1][c]$.


\begin{figure}
\sidebysidevertical
{
	\begin{tabular}{c|c|c|c|c}
	 &               &               &               & \\ \hline \rowcolor{yellow!40}
	 &               & $Dat_{r-1,c}$ &               & \\ \hline \rowcolor{green!40}
	 & $Dat_{r,c-1}$ & $Dat_{r,c}$   & $Dat_{r,c+1}$ & \\ \hline \rowcolor{red!40}
	 &               & $Dat_{r+1,c}$ &               & \\ \hline
	 &               &               &               &
	\end{tabular}
	\caption{Logical layout of the $Dat$ array. Neighbours are horizontally and vertically contiguous.}
}
{
	\begin{tabular}{c|c|c|c|c|c|c|c|c}
	\hline
	\ldots & \cellcolor{yellow!40} $Dat_{r-1,c}$ &
	\ldots & \cellcolor{green!40} $Dat_{r,c-1}$ & \cellcolor{green!40} $Dat_{r,c}$ & \cellcolor{green!40} $Dat_{r,c+1}$ &
	\ldots & \cellcolor{red!40} $Dat_{r+1,c}$ &
	\ldots \\
	\hline
	\end{tabular}
	\caption{Physical layout of the $Dat$ array. The contiguous data elements are contiguous in memory.}
}
\caption{The layout the $Dat$ array, representing vertex-vertex associative data. This data stored may represent the spatial coordinates of vertices, for example.}
\end{figure}



\section{Overlaying the remaining mesh elements}
Our construction thus far only allows us to directly address data via a vertex-vertex relation, and we would like to extend the benefits to include other relation-maps. The key insight is the knowledge that relation-maps do not define an arbitrary relation, rather they \emph{represent compositional relationships in the mesh hierarchy}:
\begin{itemize}
\item A structured vertex $n_{r,c}$ forms a horizontal edge with $n_{r,c+1}$, and a vertical edge with $n_{r+1,c}$.
\item Four structured vertices of the given relative positions form the vertices of a quadrilateral cell: $n_{r,c}$, $n_{r,c+1}$, $n_{r+1,c}$, and $n_{r+1,c+1}$.
\item Four structured vertex pairs of the given relative positions form the edges of a quadrilateral cell: $n_{r,c}$ with $n_{r,c+1}$, $n_{r,c+1}$ with $n_{r+1,c+1}$, $n_{r+1,c+1}$ with $n_{r+1,c}$, and $n_{r+1,c}$ with $n_{r,c}$.
\end{itemize}


We can then derive for each structured vertex region
\begin{enumerate*}[label=\alph*)]
\item a structured cell region,
\item a structured horizontal-edges region, and
\item a structured vertical-edge region
\end{enumerate*}.
We refer to a given structured vertex region and the structured regions derived from it as \emph{corresponding structured regions}.

The edge case is an interesting one: the reason that we derive two structured regions, horizontal and vertical, is that they exhibit a different neighbour access pattern.


In addition, as with vertex associative data, the associative data for each of the derived structured regions may be laid out in any convenient order; we continue with the same row-major order.

\subsection{Redefining boundaries}
Given the above information, we can generalize our definition of interior and fringe structured elements to allow for relation-maps between different element sets.

Let $R: S \mapsto T$ be a relation-map between element sets $S$ and $T$. We assume that $S$ and $T$ have each been partitioned into $k$ mutually \emph{corresponding} structured regions $S_1$ to $S_k$ and $T_1$ to $T_k$, respectively, where each structured region $S_i$ corresponds to structured region $T_i$. Additionally, $S_0$ and $T_0$ represent the unstructured partitions in each of $S$ and $T$\footnote{As per our definition, $S_0$ and $T_0$ cannot be corresponding structured regions, as they are not structured!}.
Let $e \in S_i$ be some structured mesh element in some structured region $S_i$.

We say that $e$ is an \emph{interior structured element} iff all neighbours $n$ of $e$, as defined by $R$, are structured mesh elements in the corresponding structured region $T_i$. Stated formally:
$$\forall n \in T. R(e,n) \implies n \in T_i$$
Conversely, a \emph{fringe structured element} is one which does not satisfy this property, that is it has some neighbour in $R$ which is not found within the corresponding structured region $T_i$.

Figure~\ref{fig:fringe-edges} shows an example for an edge-to-cell relation-map.

\begin{figure}
\includesvg[width=\imagewidth, svgpath=images/renumbering/]{fringe-edges}
\caption{The image depicts the boundaries induced by an edge-to-cell relation-map. Fringe structured edges (having only one structured cell neighbour) are highlighted in dark blue, and interior structured edges (having both neighbours structured cells) are highlighted in light blue. All structured cells are are dotted. Structured vertices, represented by circles, demarcate the structured vertex region.}
\label{fig:fringe-edges}
\end{figure}


\subsection{Identifying element numbering}
Deriving an overlaid structured region is simple, and allows for direct access to neighbouring elements. This derivation method, however, does not reveal the implicitly defined partitioning of mesh elements (into structured and unstructured mesh elements), in particular we do not know how to iterate over the unstructured mesh elements, as demonstrated in figure~\ref{fig:comparing-loops} for the case of a cell-cell relation-map.

If we can find the element ids corresponding to the derived structured elements, we can then deduce contents of the other partition: the unstructured elements. For this we require a relation-map from vertices to the element type in question, so then we can map the vertices forming the derived mesh element to its respective element id. This relation-map may already be provided, or we may need to derive it from other relation-maps.

As an example, consider a relation-map mapping each cell to a unique set of four vertices. If we invert this map, then we obtain a relation-map mapping sets of four vertices to a unique cell, each. Since the map is indexed by sets of vertices, a direct-indexed array is not an appropriate choice of data structure, and the use of, for example, a hash table would be required. Alternatively, we may represent the relation-map as a map $\AdjVC$ mapping each vertex to a set of cells adjacent to it. Determining the unique cell (if any) composed of four vertices $\{ n_1, n_2, n_3, n_4\}$ can then be determined by the following:
$$\AdjVC(n_1) \cap \AdjVC(n_2) \cap \AdjVC(n_3) \cap \AdjVC(n_4)$$

\begin{figure}
% Listing 1
\newsavebox{\firstlisting}
\begin{lrbox}{\firstlisting}

\begin{lstlisting}[language=python]
# Apply the kernel function to each cell and its neighbour cells.
for (cell_id, neighbour_cells_ids) in cell2cells:
	kernel_function(cell_id, neighbour_cells_ids)
\end{lstlisting}
\end{lrbox}

% Listing 2
\newsavebox{\secondlisting}
\begin{lrbox}{\secondlisting}
\begin{lstlisting}[language=python]
# Apply the kernel function to each interior structured cell
# and its neighbour cells.
for region in structured_regions:
	# ... details omitted
	kernel_function(cell_id, neighbour_cells_ids)

# Apply the kernel function to the remaining unvisited cells
remaining_cell2cells = ... # this is currently unknown
for (cell_id, neighbour_cells_ids) in remaining_cell2cells:
	kernel_function(cell_id, neighbour_cells_ids)
\end{lstlisting}
\end{lrbox}

% Side-by-side
\sidebysideverticalnoncenter
{
\usebox{\firstlisting}
\caption{Applying a kernel function over the original cell-cell relation-map. \lstinline|cell2cells| is the cell-cell relation-map.}
\label{subfig:original-loop}
}
{
\usebox{\secondlisting}
\caption{Applying a kernel function over the interior structured cells, followed by the remaining cells. \lstinline|remaining_cell2cells| is a subset of the cell-cell relation-map which excludes interior structured cells.}
\label{subfig:crystal-loop}
}

\caption{A comparison between ~(\ref{subfig:original-loop}) the original iteration over cells, and~(\ref{subfig:crystal-loop}) the split iterations over the interior structured cells and the remaining cells. Not knowing the iteration space of the remaining cells loop demonstrates the need to determine the structure-induced partitioning.}
\label{fig:comparing-loops}
\end{figure}



\begin{figure}
	\includesvg[width=\imagewidth, svgpath=images/renumbering/]{inverting-map}

{\begin{tabular}{|ccc|}
$c_{0}$ & $\mapsto$ & $\{n_{16},n_{15},n_{8},n_{6}\}$ \\
$c_{1}$ & $\mapsto$ & $\{n_{10},n_{19},n_{9},n_{0}\}$ \\
$c_{2}$ & $\mapsto$ & $\{n_{8},n_{6},n_{12},n_{14}\}$ \\
$c_{3}$ & $\mapsto$ & $\{n_{15},n_{3},n_{6},n_{2}\}$ \\
$c_{4}$ & $\mapsto$ & $\{n_{18},n_{7},n_{10},n_{19}\}$ \\
$c_{5}$ & $\mapsto$ & $\{n_{13},n_{13},n_{7},n_{16}\}$ \\
$c_{6}$ & $\mapsto$ & $\{n_{7},n_{16},n_{19},n_{8}\}$ \\
$c_{7}$ & $\mapsto$ & $\{n_{6},n_{2},n_{14},n_{11}\}$ \\
$c_{8}$ & $\mapsto$ & $\{n_{4},n_{5},n_{15},n_{3}\}$ \\
$c_{9}$ & $\mapsto$ & $\{n_{1},n_{17},n_{18},n_{7}\}$ \\
$c_{10}$ & $\mapsto$ & $\{n_{19},n_{8},n_{0},n_{12}\}$ \\
$c_{11}$ & $\mapsto$ & $\{n_{13},n_{4},n_{16},n_{15}\}$ \\
\end{tabular}}



{\begin{tabular}{|ccc|}
$n_{0}$ & $\mapsto$ & $\{c_{1},c_{10}\}$ \\
$n_{1}$ & $\mapsto$ & $\{c_{9}\}$ \\
$n_{2}$ & $\mapsto$ & $\{c_{3},c_{7}\}$ \\
$n_{3}$ & $\mapsto$ & $\{c_{8},c_{3}\}$ \\
$n_{4}$ & $\mapsto$ & $\{c_{8},c_{11}\}$ \\
$n_{5}$ & $\mapsto$ & $\{c_{8}\}$ \\
$n_{6}$ & $\mapsto$ & $\{c_{0},c_{2},c_{3},c_{7}\}$ \\
$n_{7}$ & $\mapsto$ & $\{c_{9},c_{4},c_{5},c_{6}\}$ \\
$n_{8}$ & $\mapsto$ & $\{c_{0},c_{2},c_{10},c_{6}\}$ \\
$n_{9}$ & $\mapsto$ & $\{c_{1}\}$ \\
$n_{10}$ & $\mapsto$ & $\{c_{1},c_{4}\}$ \\
$n_{11}$ & $\mapsto$ & $\{c_{7}\}$ \\
$n_{12}$ & $\mapsto$ & $\{c_{2},c_{10}\}$ \\
$n_{13}$ & $\mapsto$ & $\{c_{11},c_{5}\}$ \\
$n_{14}$ & $\mapsto$ & $\{c_{2},c_{7}\}$ \\
$n_{15}$ & $\mapsto$ & $\{c_{0},c_{8},c_{3},c_{11}\}$ \\
$n_{16}$ & $\mapsto$ & $\{c_{0},c_{11},c_{5},c_{6}\}$ \\
$n_{17}$ & $\mapsto$ & $\{c_{9}\}$ \\
$n_{18}$ & $\mapsto$ & $\{c_{9},c_{4}\}$ \\
$n_{19}$ & $\mapsto$ & $\{c_{1},c_{10},c_{4},c_{6}\}$ \\
\end{tabular}}



{\begin{tabular}{|ccc|}
$\{n_{16},n_{15},n_{8},n_{6}\}$ & $\mapsto$ & $c_{0}$ \\
$\{n_{10},n_{19},n_{9},n_{0}\}$ & $\mapsto$ & $c_{1}$ \\
$\{n_{8},n_{6},n_{12},n_{14}\}$ & $\mapsto$ & $c_{2}$ \\
$\{n_{15},n_{3},n_{6},n_{2}\}$ & $\mapsto$ & $c_{3}$ \\
$\{n_{18},n_{7},n_{10},n_{19}\}$ & $\mapsto$ & $c_{4}$ \\
$\{n_{13},n_{13},n_{7},n_{16}\}$ & $\mapsto$ & $c_{5}$ \\
$\{n_{7},n_{16},n_{19},n_{8}\}$ & $\mapsto$ & $c_{6}$ \\
$\{n_{6},n_{2},n_{14},n_{11}\}$ & $\mapsto$ & $c_{7}$ \\
$\{n_{4},n_{5},n_{15},n_{3}\}$ & $\mapsto$ & $c_{8}$ \\
$\{n_{1},n_{17},n_{18},n_{7}\}$ & $\mapsto$ & $c_{9}$ \\
$\{n_{19},n_{8},n_{0},n_{12}\}$ & $\mapsto$ & $c_{10}$ \\
$\{n_{13},n_{4},n_{16},n_{15}\}$ & $\mapsto$ & $c_{11}$ \\
\end{tabular}}
	% {
	% 	\begin{tabular}{|c|c|c|c|c|c|c|c|c|c|c|c|}
	% 	0                & 1               & 2               & 3               & 4                & 5                &
	% 	6                & 7               & 8               & 9               & 10               & 11               \\
	% 	$\{16,15,8,6\}$  & $\{10,19,9,0\}$ & $\{8,6,12,14\}$ & $\{15,3,6,2\}$  & $\{18,7,10,19\}$ & $\{13,13,7,16\}$ &
	% 	$\{7,16,19,8\}$  & $\{6,2,14,11\}$ & $\{4,5,15,3\}$  & $\{1,17,18,7\}$ & $\{19,8,0,12\}$  & $\{13,4,16,15\}$ \\
	% 	\end{tabular}
	% }

	% {
	% 	\begin{tabular}{|c|c|c|c|c|c|c|c|c|c|c|c|c|c|c|c|c|c|c|c|}
	% 	0 & 1 & 2 & 3 & 4 & 5 & 6 & 7 & 8 & 9 & 10 & 11 & 12 & 13 & 14 & 15 & 16 & 17 & 18 & 19 \\
	% 	$\{1,10\}$ & $\{9\}$ & $\{3,7\}$ & $\{8,3\}$ & $\{8,11\}$ & $\{8\}$ & $\{0,2,3,7\}$ & $\{9,4,5,6\}$ & $\{0,2,10,6\}$ & $\{1\}$ & $\{1,4\}$ & $\{7\}$ & $\{2,10\}$ & $\{11,5\}$ & $\{2,7\}$ & $\{0,8,3,11\}$ & $\{0,11,5,6\}$ & $\{9\}$ & $\{9,4\}$ & $\{1,10,4,6\}$
	% 	\end{tabular}
	% }

	% {
	% 	\begin{tabular}{|c|c|c|c|c|c|c|c|c|c|c|c|c|c|c|c|c|c|c|c|}
	% 	$\{16,15,8,6\}$ & $\{10,19,9,0\}$ & $\{8,6,12,14\}$ & $\{15,31,6,2\}$ & $\{18,7,10,19\}$ & $\{13,13,7,16\}$ & $\{7,16,19,8\}$ & $\{6,2,14,11\}$ & $\{4,5,15,3\}$ & $\{1,17,18,7\}$ & $\{19,8,0,12\}$ & $\{13,4,16,15\}$ \\
	% 	0 & 1 & 2 & 3 & 4 & 5 & 6 & 7 & 8 & 9 & 10 & 11
	% 	\end{tabular}
	% }

	\caption{}
\end{figure}


% \begin{lstlisting}[language=python]
% for region in structured_regions:

% 	# For each interior cell in region
% 	for r in (1..region.num_rows-1):
% 		for c in (1..region.num_cols-1):
% 			cell_id = region.offset + r * region.elems_per_column + c
% 			cell_north = region.offset + (r-1) * region.elems_per_column + c
% 			cell_east = region.offset + r * region.elems_per_column + (c+1)
% 			cell_south = region.offset + (r-1) * region.elems_per_column + c
% 			cell_east = region.offset + r * region.elems_per_column + (c+1)


% for (cell_id, neighbour_cells_ids) in cell2cells:
% 	kernel_function(cell_id, neighbour_cells_ids)
% \end{lstlisting}


involved, in particular the fringe structured elements and the unstructured elements.

We need to identify the elements in the overlaid structured region, as we have with vertices, which in turn will identify the other partition containing the fringe structured elements and the unstructured elements.

The derived mesh elements need to be associated with a mesh entity, by obtaining its unique identifier.









Now iterating over \emph{interior} structured mesh elements no longer requires a relation-map: for a given element we can derive any of its \emph{structured} neighbours of any element type in a stencil-like fashion (as explained in subsection~\ref{subsec:given-kernel-function}).

%% TODO figure showing how to get from cell id to node ids , edge id to nodes/cells, etc






\section{Renumbering elements}
So far we have only considered neighbour accesses for interior structured elements; this leaves us with two classes of mesh elements:
\begin{itemize}
\item Fringe structured mesh elements.
These may have both structured and unstructured neighbours (or potentially non-neighbours if the element is at a \emph{mesh} boundary). They are accessed directly via address calculation by their interior structured neighbours, and indirectly via relation-maps by their fringe structured neighbours and their unstructured neighbours (if any).

\item Unstructured mesh elements.
These always access (and are accessed by) their neighbours indirectly via relation-maps.
\end{itemize}

Observe that mesh elements that are accessed directly, the structured elements in other words, have an imposed storage location. This is the reason that structured regions must be disjoint, otherwise multiple structured region would impose conflicting storage locations on the same element\footnote{Whilst duplicating data would work for read-only access, at the expense of extra storage, this can be a disaster for writeable data as duplicate data must be correctly synchronized.}. Mesh elements that are indirectly accessed on the other hand are not constrained in their placement. Thus we can treat fringe structured mesh elements as structured in terms of their storage placement, and as unstructured when accessing their neighbours.

%% TODO figure with different accesses
We can enumerate all the possible neighbour access types as follows:
\begin{itemize}
\item Two interior structured elements.
Both elements must be co-located in the same structured region, and are directly addressable. No further data needs to be stored.
\item An interior structured element and a fringe structured element.
Same as above.
\item Two fringe structured elements.
The two elements may or may not fall in the same structured region. In the former case, the above applies. In the latter case, why may occur if two structured regions are adjacent, the elements must be indirectly addressed.
\item A fringe structured element and an unstructured element.
The two elements are not co-located and must be indirectly addressed.
\item Two unstructured elements.
The two elements are co-located in the same partition, but must be indirectly addressed.
\end{itemize}

We would like to minimize the cost and complication of cross-region neighbour access, which arise due to fringe structured elements. To this avail, all regions for the given associative data, structured and unstructured, are stored in a single address space. All structured regions are stored sequentially in the order of their discovery\footnote{This is no compelling reason for this decision; this was merely done to simplify adding structured regions as they are discovered.}, followed by the remaining unstructured elements.
The relation-map can then simply store indices to denote each element's neighbours. In fact, if we maintain the explicit neighbour relations for \emph{all} mesh elements, both structured and unstructured, a core-computation loop completely oblivious to our manipulations may execute over the relation-maps.

% TODO FUTURE WORK?
The main downside to storing is the missed opportunity to completely exclude interior structured nodes from the relation map.


Figures~\ref{fig:original-numbering} and~\ref{fig:renumbering} show an example of renumbering applied to cells.

% Graph colors
\definecolor{red-interior}{RGB}{255,157,157}
\definecolor{red-fringe}{RGB}{255,0,0}
\definecolor{blue-interior}{RGB}{0,255,255}
\definecolor{blue-fringe}{RGB}{0,194,194}
\newlength{\charheight}
\settoheight{\charheight}{\hbox{A}}

\begin{figure}
\sidebysidevertical
{
	\includesvg[width=\imagewidth, svgpath=images/renumbering/]{fringe-cells-original-numbers}
	\caption{Mesh showing the the original cell numbering of the structured regions. Numbering of unstructured cells is omitted. Light and dark shades denote interior and fringe structured cells, respectively.}
}
{
	\small
	\tabcolsep=0.32mm
	\begin{tabular}{|cccccccccccccccccccc|cccccccccccccccccccc|cccccccccccccccccccc|cccccccccccccccccccc|cccccccccccccccccccc|cccccccccccccccccccc|cccccccccccccccccccc|cccccccccccccccccccc|cccccccccccccccccccc|cccccccccccccccccccc|}
\cellcolor{white} & \cellcolor{white} & \cellcolor{blue-interior} & \cellcolor{blue-fringe} & \cellcolor{blue-interior} & \cellcolor{blue-fringe} & \cellcolor{blue-interior} & \cellcolor{blue-interior} & \cellcolor{white} & \cellcolor{white} & \cellcolor{white} & \cellcolor{blue-interior} & \cellcolor{white} & \cellcolor{white} & \cellcolor{white} & \cellcolor{white} & \cellcolor{white} & \cellcolor{white} & \cellcolor{white} & \cellcolor{red-interior} & \cellcolor{red-interior} & \cellcolor{white} & \cellcolor{white} & \cellcolor{white} & \cellcolor{white} & \cellcolor{white} & \cellcolor{white} & \cellcolor{blue-interior} & \cellcolor{white} & \cellcolor{white} & \cellcolor{blue-fringe} & \cellcolor{white} & \cellcolor{white} & \cellcolor{red-fringe} & \cellcolor{blue-interior} & \cellcolor{blue-interior} & \cellcolor{blue-interior} & \cellcolor{red-interior} & \cellcolor{red-interior} & \cellcolor{white} & \cellcolor{white} & \cellcolor{white} & \cellcolor{white} & \cellcolor{red-fringe} & \cellcolor{blue-interior} & \cellcolor{blue-interior} & \cellcolor{white} & \cellcolor{white} & \cellcolor{white} & \cellcolor{blue-fringe} & \cellcolor{white} & \cellcolor{blue-interior} & \cellcolor{white} & \cellcolor{blue-fringe} & \cellcolor{blue-interior} & \cellcolor{blue-interior} & \cellcolor{white} & \cellcolor{white} & \cellcolor{white} & \cellcolor{white} & \cellcolor{white} & \cellcolor{white} & \cellcolor{white} & \cellcolor{white} & \cellcolor{white} & \cellcolor{blue-fringe} & \cellcolor{blue-fringe} & \cellcolor{red-interior} & \cellcolor{white} & \cellcolor{blue-interior} & \cellcolor{white} & \cellcolor{white} & \cellcolor{white} & \cellcolor{white} & \cellcolor{blue-interior} & \cellcolor{white} & \cellcolor{white} & \cellcolor{white} & \cellcolor{white} & \cellcolor{white} & \cellcolor{blue-interior} & \cellcolor{white} & \cellcolor{blue-fringe} & \cellcolor{white} & \cellcolor{white} & \cellcolor{white} & \cellcolor{blue-fringe} & \cellcolor{white} & \cellcolor{white} & \cellcolor{white} & \cellcolor{white} & \cellcolor{white} & \cellcolor{white} & \cellcolor{blue-interior} & \cellcolor{blue-interior} & \cellcolor{white} & \cellcolor{white} & \cellcolor{white} & \cellcolor{white} & \cellcolor{blue-interior} & \cellcolor{red-interior} & \cellcolor{white} & \cellcolor{white} & \cellcolor{white} & \cellcolor{white} & \cellcolor{blue-interior} & \cellcolor{blue-fringe} & \cellcolor{white} & \cellcolor{blue-interior} & \cellcolor{white} & \cellcolor{white} & \cellcolor{white} & \cellcolor{white} & \cellcolor{blue-fringe} & \cellcolor{white} & \cellcolor{blue-fringe} & \cellcolor{white} & \cellcolor{blue-interior} & \cellcolor{blue-interior} & \cellcolor{blue-fringe} & \cellcolor{white} & \cellcolor{white} & \cellcolor{white} & \cellcolor{blue-interior} & \cellcolor{white} & \cellcolor{blue-interior} & \cellcolor{blue-fringe} & \cellcolor{white} & \cellcolor{white} & \cellcolor{blue-interior} & \cellcolor{white} & \cellcolor{white} & \cellcolor{white} & \cellcolor{red-interior} & \cellcolor{blue-interior} & \cellcolor{white} & \cellcolor{blue-fringe} & \cellcolor{white} & \cellcolor{white} & \cellcolor{blue-interior} & \cellcolor{white} & \cellcolor{white} & \cellcolor{white} & \cellcolor{blue-fringe} & \cellcolor{blue-fringe} & \cellcolor{blue-fringe} & \cellcolor{white} & \cellcolor{blue-fringe} & \cellcolor{white} & \cellcolor{blue-interior} & \cellcolor{blue-fringe} & \cellcolor{white} & \cellcolor{blue-fringe} & \cellcolor{blue-fringe} & \cellcolor{white} & \cellcolor{blue-fringe} & \cellcolor{white} & \cellcolor{white} & \cellcolor{white} & \cellcolor{blue-interior} & \cellcolor{blue-fringe} & \cellcolor{blue-interior} & \cellcolor{white} & \cellcolor{blue-interior} & \cellcolor{blue-interior} & \cellcolor{white} & \cellcolor{white} & \cellcolor{blue-interior} & \cellcolor{blue-fringe} & \cellcolor{white} & \cellcolor{blue-interior} & \cellcolor{white} & \cellcolor{white} & \cellcolor{white} & \cellcolor{white} & \cellcolor{blue-fringe} & \cellcolor{blue-interior} & \cellcolor{blue-interior} & \cellcolor{blue-interior} & \cellcolor{blue-interior} & \cellcolor{blue-fringe} & \cellcolor{blue-interior} & \cellcolor{blue-fringe} & \cellcolor{white} & \cellcolor{blue-interior} & \cellcolor{white} & \cellcolor{white} & \cellcolor{blue-fringe} & \cellcolor{white} & \cellcolor{white} & \cellcolor{white} & \cellcolor{white} & \cellcolor{white} & \cellcolor{white} & \cellcolor{blue-fringe} & \cellcolor{red-interior} & \cellcolor{red-interior} & \cellcolor{white} & \cellcolor{white} & \cellcolor{red-interior} \\ [5ex]
\multicolumn{20}{|l|}{0} &
\multicolumn{20}{l|}{20} & \multicolumn{20}{l|}{40} & \multicolumn{20}{l|}{60} & \multicolumn{20}{l|}{80} & \multicolumn{20}{l|}{100} & \multicolumn{20}{l|}{120} & \multicolumn{20}{l|}{140} & \multicolumn{20}{l|}{160} & \multicolumn{20}{l|}{180} \\
\end{tabular}

	\caption{The original data layout of cell associative data.}
}
\caption{The original cell numbering. The boundaries shown are with respect to a \emph{cell-cell} relation-map.}
\label{fig:original-numbering}
\end{figure}


\begin{figure}
\sidebysidevertical
{
	\includesvg[width=\imagewidth, svgpath=images/renumbering/]{fringe-cells-renumbered}
	\caption{A mesh with the Crystal cell numbering shown. Numbering of unstructured cells is omitted. Light and dark shades denote interior and fringe structured cells, respectively.}
}
{
	\small
	\tabcolsep=0.32mm
	\begin{tabular}{|cccccccccccccccccccc|cccccccccccccccccccc|cccccccccccccccccccc|cccccccccccccccccccc|cccccccccccccccccccc|cccccccccccccccccccc|cccccccccccccccccccc|}
\cellcolor{red-fringe} & \cellcolor{red-fringe} & \cellcolor{red-fringe} & \cellcolor{red-interior} & \cellcolor{red-fringe} & \cellcolor{red-fringe} & \cellcolor{red-interior} & \cellcolor{red-fringe} & \cellcolor{red-fringe} & \cellcolor{red-fringe} & \cellcolor{red-fringe} & \cellcolor{red-fringe} & \cellcolor{blue-fringe} & \cellcolor{blue-fringe} & \cellcolor{blue-fringe} & \cellcolor{blue-fringe} & \cellcolor{blue-fringe} & \cellcolor{blue-fringe} & \cellcolor{blue-fringe} & \cellcolor{blue-fringe} & \cellcolor{blue-fringe} & \cellcolor{blue-fringe} & \cellcolor{blue-interior} & \cellcolor{blue-interior} & \cellcolor{blue-interior} & \cellcolor{blue-interior} & \cellcolor{blue-interior} & \cellcolor{blue-interior} & \cellcolor{blue-interior} & \cellcolor{blue-fringe} & \cellcolor{blue-fringe} & \cellcolor{blue-interior} & \cellcolor{blue-interior} & \cellcolor{blue-interior} & \cellcolor{blue-interior} & \cellcolor{blue-interior} & \cellcolor{blue-interior} & \cellcolor{blue-interior} & \cellcolor{blue-fringe} & \cellcolor{blue-fringe} & \cellcolor{blue-interior} & \cellcolor{blue-interior} & \cellcolor{blue-interior} & \cellcolor{blue-interior} & \cellcolor{blue-interior} & \cellcolor{blue-interior} & \cellcolor{blue-interior} & \cellcolor{blue-fringe} & \cellcolor{blue-fringe} & \cellcolor{blue-interior} & \cellcolor{blue-interior} & \cellcolor{blue-interior} & \cellcolor{blue-interior} & \cellcolor{blue-interior} & \cellcolor{blue-interior} & \cellcolor{blue-interior} & \cellcolor{blue-fringe} & \cellcolor{blue-fringe} & \cellcolor{blue-interior} & \cellcolor{blue-interior} & \cellcolor{blue-interior} & \cellcolor{blue-interior} & \cellcolor{blue-interior} & \cellcolor{blue-interior} & \cellcolor{blue-interior} & \cellcolor{blue-fringe} & \cellcolor{blue-fringe} & \cellcolor{blue-interior} & \cellcolor{blue-interior} & \cellcolor{blue-interior} & \cellcolor{blue-interior} & \cellcolor{blue-interior} & \cellcolor{blue-interior} & \cellcolor{blue-interior} & \cellcolor{blue-fringe} & \cellcolor{blue-fringe} & \cellcolor{blue-fringe} & \cellcolor{blue-fringe} & \cellcolor{blue-fringe} & \cellcolor{blue-fringe} & \cellcolor{blue-fringe} & \cellcolor{blue-fringe} & \cellcolor{blue-fringe} & \cellcolor{blue-fringe} & \cellcolor{white} & \cellcolor{white} & \cellcolor{white} & \cellcolor{white} & \cellcolor{white} & \cellcolor{white} & \cellcolor{white} & \cellcolor{white} & \cellcolor{white} & \cellcolor{white} & \cellcolor{white} & \cellcolor{white} & \cellcolor{white} & \cellcolor{white} & \cellcolor{white} & \cellcolor{white} & \cellcolor{white} & \cellcolor{white} & \cellcolor{white} & \cellcolor{white} & \cellcolor{white} & \cellcolor{white} & \cellcolor{white} & \cellcolor{white} & \cellcolor{white} & \cellcolor{white} & \cellcolor{white} & \cellcolor{white} & \cellcolor{white} & \cellcolor{white} & \cellcolor{white} & \cellcolor{white} & \cellcolor{white} & \cellcolor{white} & \cellcolor{white} & \cellcolor{white} & \cellcolor{white} & \cellcolor{white} & \cellcolor{white} & \cellcolor{white} & \cellcolor{white} & \cellcolor{white} & \cellcolor{white} & \cellcolor{white} & \cellcolor{white} & \cellcolor{white} & \cellcolor{white} & \cellcolor{white} & \cellcolor{white} & \cellcolor{white} & \cellcolor{white} & \cellcolor{white} & \cellcolor{white} & \cellcolor{white} & \cellcolor{white} & \cellcolor{white} \\ [5ex]
\multicolumn{20}{|l|}{0} & 
\multicolumn{20}{l|}{20} & \multicolumn{20}{l|}{40} & \multicolumn{20}{l|}{60} & \multicolumn{20}{l|}{80} & \multicolumn{20}{l|}{100} & \multicolumn{20}{l|}{120} \\
\end{tabular}

	\caption{The Crystal data layout of cell associative data.}
}
\caption{Crystal cell numbering. The boundaries shown are with respect to a \emph{cell-cell} relation-map.}
\label{fig:renumbering}
\end{figure}


Looking back to the definition of a kernel function in subsection~\ref{subsec:given-kernel-function}, we find that in addition to the derived indexing variables of neighbours, the kernel function also requires the indexing variable referring to the current element.




\subsection{Accessing neighbours in-order}
the indexing variables, however, are passed to the kernel in some known order, typically in the order stored in the relation-map
At present, we are able to

