\newcommand{\Structured}{V_{structured}}
\newcommand{\AdjVV}{Adj_{\VertexSet\VertexSet}}
\newcommand{\vinit}{v_{init}}



\subsection{Inputs}
\begin{enumerate}
\item A non-reflexive and symmetric vertex-vertex adjacency relation:
$$ \AdjVV: \VertexSet \mapsto \powerset{\VertexSet} $$

\item A set of visited vertices
$$ \Visited \subseteq \VertexSet $$

\item An unvisited start vertex
$$ \vinit \in \VertexSet \setminus \Visited $$
\end{enumerate}


\subsection{Outputs}
\begin{enumerate}
\item A structured set of vertices $\Structured \subseteq \VertexSet \setminus \Visited $ forming the extracted structured region.
The vertices $\Structured$ are structured on a 2-dimensional Cartesian lattice with $m$ rows and $n$ columns.
\end{enumerate}


%%%% PHASE 1

%% Phase 1 summary
\subsection{Phase 1: Grow a quad}
Starting from the initial vertex $\vinit$, call it $n_{1,1}$, we would like to discover three other vertices $n_{1,2}$, $n_{2,1}$, and $n_{2,2}$ that, together with $n_{1,1}$, form a quad in a structured quad region. They must satisfy the following constraints:

\begin{itemize}
\item Each of the four vertices must have exactly 4 neighbours.
\item Each of the following pairs of vertices are neighbours: $n_{1,1}$~and~$n_{1,2}$ ; $n_{1,2}$~and~$n_{2,2}$ ; $n_{2,2}$~and~$n_{2,1}$ ; $n_{2,1}$~and~$n_{1,1}$.
\item Each of the vertices must \emph{not} neighbour any vertex that is part of the structured region thus far, apart from those vertices explicitly mentioned.
\end{itemize}

%% Phase 1 diagram

\begin{tikzpicture}[scale=1]
		% Default action for each node
		\tikzstyle{every node}=[draw, shape=circle, minimum size=1cm];

											\coordinate (northleft) at (0, 2);	\coordinate (northright) at (2, 2);
		\coordinate (westup) at (-1,1);		\node (r1c1) at (0,1) {$n_{1,1}$};	\node (r1c2) at (2,1) {$n_{1,2}$};	\coordinate (eastup) at (3,1);
		\coordinate (westdown) at (-1,-1);	\node (r2c1) at (0,-1) {$n_{2,1}$};	\node (r2c2) at (2,-1) {$n_{2,2}$};	\coordinate (eastdown) at (3,-1);
											\coordinate (southleft) at (0, -2);	\coordinate (southright) at (2, -2);

		% Horizontals
		\draw[outside] (westup) -- (r1c1);
		\draw (r1c1) -- (r1c2);
		\draw[outside] (r1c2) -- (eastup);

		\draw[outside] (westdown) -- (r2c1);
		\draw (r2c1) -- (r2c2);
		\draw[outside] (r2c2) -- (eastdown);

		% Verticals
		\draw[outside] (northleft) -- (r1c1);
		\draw (r1c1) -- (r2c1);
		\draw[outside] (r2c1) -- (southleft);

		\draw[outside] (northright) -- (r1c2);
		\draw (r1c2) -- (r2c2);
		\draw[outside] (r2c2) -- (southright);
	\end{tikzpicture}

%% Phase 1 algorithm

\subsubsection{Algorithm}

\begin{enumerate}
\item If $\vinit$ does not have exactly 4 neighbours, that is $ \card{\AdjVV(\vinit)} \neq 4 $, return immediately with $\Structured = \varnothing$.
\item Let $ n_{1,1} = \vinit $, and let $ a, b, c \in \AdjVV(\vinit) $ be distinct vertex neighbours of $\vinit$. Consider vertices $a$ and $b$. If they do not both have exactly 4 neighbours, then they cannot form a part of a structured quad region. Return immediately with $\Structured = \varnothing$.

\item Otherwise, there are three cases:

	%%%% BEGIN THREE CASES
	\begin{enumerate}[label=Case \alph*)]

	%% Straight line case
	\item $a$ and $b$ have exactly one neighbour in common, which must be $n_{1,1}$ by construction, expressed by:
	$$ \card{\AdjVV(a) \cap \AdjVV(b)} = 1$$
	$a, n_{1,1}, b $ are \emph{topologically} along a straight line of a structured grid, and hence cannot form a quad.
	We therefore consider $b$, $n_{1,1}$, and  $c$ instead as candidates, letting $n_{2,1} = b$ and $n_{1,2} = c$.
	\begin{tikzpicture}
		% Default action for each node
		\tikzstyle{every node}=[draw, shape=circle, minimum size=1.3cm];

										\coordinate (northleft) at (-1,2);	\node (b) at (1,3) {$b / n_{2,1}$};	\coordinate (northright) at (3,2);
		\coordinate (west) at (-2,1);	\node (a) at (-1,1) {$a$};			\node (n11) at (1,1) {$n_{1,1}$};	\node (c) at (3,1) {$c / n_{1,2}$};	\coordinate (east) at (4,1);
										\coordinate (southleft) at (-1,0);	\coordinate (southmid) at (1,0);	\coordinate (southright) at (3,0);

		\draw[anywhere] (west) -- (a);
		\draw (a) -- (n11) -- (c);
		\draw[anywhere] (c) -- (east);

		\draw[anywhere] (northleft) -- (a) -- (southleft);
		\draw (b) -- (n11);
		\draw[outside] (n11) -- (southmid);
		\draw[anywhere] (northright) -- (c) -- (southright);
	\end{tikzpicture}


	%% Angle case
	\item $a$ and $b$ have exactly two neighbours in common, one of which must be $n_{1,1}$ by construction, expressed by:
	$$ \card{\AdjVV(a) \cap \AdjVV(b)} = 2$$
	We continue with vertices $a$, $n_{1,1}$, and $b$ as candidates, letting $n_{2,1} = a$ and $n_{1,2} = b$.
	\begin{tikzpicture}
		% Default action for each node
		\tikzstyle{every node}=[draw, shape=circle, minimum size=1.3cm];

											\coordinate (northleft) at (-1,4);		\coordinate (northright) at (1,4);
		\coordinate (westup) at (-2,3);		\node (a) at (-1,3) {$a / n_{2,1}$};	\node (d) at (1,3) {$?$};			\coordinate (eastup) at (2,3);
		\coordinate (westdown) at (-2,1);	\node (n11) at (-1,1) {$n_{1,1}$};		\node (b) at (1,1) {$b / n_{1,2}$};	\coordinate (eastdown) at (2,1);
											\coordinate (southleft) at (-1,0);		\coordinate (southright) at (1,0);



		\draw[outside] (westup) -- (a);
		\draw (a) -- (d);
		% \draw (d) -- (eastup);

		\draw[anywhere] (westdown) -- (n11);
		\draw (n11) -- (b);
		\draw[outside] (b) -- (eastdown);

		\draw[outside] (northleft) -- (a);
		\draw (a) -- (n11);
		\draw[anywhere] (n11) -- (southleft);

		% \draw (northright) -- (d);
		\draw (d) -- (b);
		\draw[outside] (b) -- (southright);

	\end{tikzpicture}


	\item $a$ and $b$ have more than two neighbours in common\footnote{note that the set is non-empty by construction}, expressed by:
	$$ \card{\AdjVV(a) \cap \AdjVV(b)} > 2$$
	We cannot form a quad, and hence return immediately with $\Structured = \varnothing$.

	\end{enumerate}
	%%%% END THREE CASES

\item Find the common neighbours of $n_{2,1}$ and $n_{1,2}$. If there are not exactly two neighbours, return immediately with $\Structured = \varnothing$.

\item One of the two neighbours must be $n_{1,1}$ by construction. Let the other neighbour be $n_{2,2}$

\item Let $N = \{ n_{1,1}, n_{1,2}, n_{2,1}, n_{2,2} \}$. If any vertex $n \in N$ is in $\Visited$, that is $N \cap \Visited \neq \varnothing$, then return immediately with $\Structured = \varnothing$. Otherwise add the vertices in $N$ to $\Visited$.

\item Ensure for every vertex $n \in N$ that its visited neighbours, $\AdjVV(n) \cap \Visited$, are exactly those explicitly stated above. If this is not the case, remove $N$ from $\Visited$ and return immediately with $\Structured = \varnothing$.

\item Set $\Structured = \{ n_{1,1}, n_{1,2}, n_{2,1}, n_{2,2} \}$, and continue to the next phase.

The structured region looks as follows thus far.
\begin{tikzpicture}
	% Default action for each node
	\tikzstyle{every node}=[draw, shape=circle, minimum size=1cm];

										\coordinate (northleft) at (-1,4);	\coordinate (northright) at (1,4);
	\coordinate (westup) at (-2,3);		\node (n21) at (-1,3) {$n_{2,1}$};	\node (n22) at (1,3) {$n_{2,2}$};	\coordinate (eastup) at (2,3);
	\coordinate (westdown) at (-2,1);	\node (n11) at (-1,1) {$n_{1,1}$};	\node (n12) at (1,1) {$n_{1,2}$};	\coordinate (eastdown) at (2,1);
										\coordinate (southleft) at (-1,0);	\coordinate (southright) at (1,0);



	\draw[outside] (westup) -- (n21);
	\draw (n21) -- (n22);
	\draw[outside] (n22) -- (eastup);

	\draw[outside] (westdown) -- (n11);
	\draw (n11) -- (n12);
	\draw[outside] (n12) -- (eastdown);

	\draw[outside] (northleft) -- (n21);
	\draw (n21) -- (n11);
	\draw[outside] (n11) -- (southleft);

	\draw[outside] (northright) -- (n22);
	\draw (n22) -- (n12);
	\draw[outside] (n12) -- (southright);

\end{tikzpicture}

\end{enumerate}
