\label{chap:implementation}
We discuss our implementation of the detection algorithms discussed previously, as well as the core-computation for the airfoil computation.

\section{Mesh storage format}
All information about a particular mesh is stored in \texttt{*.p} and \texttt{*.p.part} files, which are formatted as ZIP archives. The various mesh components are stored in separate files within the archive.

\subsection{\texttt{*.p.part} file}
\label{subsec:p-part}
A \texttt{*.p.part} file stores the basic mesh information which do not related to structure. Unstructured core-computations (which only use the relation-maps) can be executed on these files. The contents are as follows:
\begin{itemize}
\item The number of elements in each entity set.
\item Associated data, typically vertex-associated spatial coordinates.
\item Relation-maps, for example an edge-cell relation-map.
\item A vertex-vertex relation-map derived from existing relation-maps. This is used for structure detection later on.
\end{itemize}

\subsection{\texttt{*.p} file}
A \texttt{*.p} file is merely a \texttt{*.p.part} with additional structure-related data appended to it. The contents include the following:
\begin{itemize}
\item Structured region meta-data, as discussed in chapter~\ref{chap:exploit-structure},
\item Renumbered associated data and relation-maps, as discussed in chapter~\ref{chap:recharting-maps}
\item Renumbering maps of the mesh entity sets, useful for debugging, visualisation, and analysis.
\end{itemize}



\section{Building \texttt{*.p.part} files}
The mesh builder is written in C++, and uses the Minizip library~\cite{minizip} for creating the archive and adding files to it. The individual components of the mesh, as described in subsection~\ref{subsec:p-part}, are created in a custom format which uses Protocol Buffers, Google's open source serialization library~\cite{protocolbuffers}.

\texttt{*.p.part} files can be built from two sources:
\begin{itemize}
\item \texttt{*.dat} files, a custom format generated by OP2~\cite{airfoilgen}.

Example: \texttt{bin/run-builder build-from naca0012.dat naca0012.p.part}
\item \texttt{*.msh} files, generated by by Gmsh, an automatic 3D finite element mesh generator~\cite{geuzaine2008gmsh}.

Example: \texttt{bin/run-builder build-from-msh naca0012.msh naca0012.p.part}
\end{itemize}


\section{Structure detection}
Structure detection is implemented entirely using Python~\cite{python}.

Python, amongst other things being an interpreted language, is inherently slow. In addition, it was targeted as a rapidly changing prototype implementation, and as such was not written with performance in mind. For this reason, we cannot fairly compare its performance to that of the core-computation, an application written in optimised C++ code.

Structure detection is split amongst three (self-evidently named) files:
\begin{itemize}
\item \texttt{detect\_node\_structure.py} defines the class \lstinline|DetectNodeStructure|. It implements the detection of multiple structure vertex regions, using the class \lstinline|DetectQuadStructure| defined in \texttt{quad\_mesh.py} to detect individual regions.
\item \texttt{detect\_cell\_structure.py} defines the class \lstinline|CellStructureFromNodeStructure|, which detects overlaid cell structured regions given the detected vertex structured regions.
\item \texttt{detect\_edge\_structure.py} defines the class \lstinline|EdgeStructureFromNodeStructure|, which similarly detects overlaid edge structured regions given the detected vertex structured regions.
\end{itemize}

\section{Building \texttt{*.p} files}
A \texttt{*.p} file can be built using the functions defined in \texttt{write\_structure\_info.py}:
\begin{itemize}
\item The function \lstinline|write_structured_node_info| writes to file the detected vertex structured regions and their associated set of relation-maps and associated data.
\item The function \lstinline|write_structured_cell_info| writes the respective cell structure region information.
\item The function \lstinline|write_structured_edge_info| writes the respective edge structure region information.
\end{itemize}

The ZIP archive files are read from and written to using the \lstinline|zipfile| module in Python's standard library.


\section{Running structure detection}
There are two python scripts which can be used to run structure detection and generate a \texttt{*.p} file :
\begin{enumerate*}[label=\alph*)]
\item \texttt{detect\_and\_append\_structure.py} detects a single vertex region, overlaid with a cell structured region and an edge structured region.
\item \texttt{detect\_multiple\_structure.py} detects multiple vertex regions by random seed-vertex sampling. Each is overlaid with a cell structured region and an edge structured region.
\end{enumerate*}


\section{Using structure in core-computation}
This is the code which facilitates the use of the detected structured regions.

\subsection{Cells kernel}
\texttt{cell\_computation.h} defines a template function \lstinline|run_structured_cells|, which can execute any kernel function over a cell-vertex relation-map.
\begin{lstlisting}[language=c++]
template<typename InArgs, typename OutArgs>
static inline void run_structured_cells(
  const QuadCellMesh& mesh,
  void cell_kernel(const int cell_id, const QuadNeighbours& vertex_neighbours, const QuadCellMesh&, const InArgs&, const OutArgs&),
  const InArgs& inputs,
  const OutArgs& outputs);
\end{lstlisting}

The function takes the following arguments:
\begin{itemize}
\item a \lstinline|QuadCellMesh| object, through which all mesh information, save for associated data, is accessed: relation-maps, structured region meta-data, and so on. This is forwarded along to the kernel function.
\item a kernel function \lstinline|cell_kernel| which takes as arguments a cell id, the ids of its vertices, the \lstinline|QuadCellMesh| object, and other arguments to be explained momentarily.
\item An argument of arbitrary type \lstinline|InArgs|, which contains associated data with read-only access. This is forwarded along to the kernel function.
\item An argument of arbitrary type \lstinline|OutArgs|, which contains associated data with read-write access. This is forwarded along to the kernel function.
\end{itemize}


\subsection{Edges kernel}
\texttt{edge\_computation.h} defines a template function \lstinline|run_structured_edges|, which can execute any kernel function over edge-vertex and edge-cell relation-maps.

\begin{lstlisting}[language=c++]
template<typename InArgs, typename OutArgs>
static inline void run_structured_edges(const QuadCellMesh& mesh,
  void edge_kernel(const int,const int, const int, const int, const QuadNeighbours&, const QuadNeighbours&, const int, const QuadCellMesh&, const InArgs&, const OutArgs&),
  const InArgs& inputs, const OutArgs& outputs)
\end{lstlisting}

The function takes similar arguments to \lstinline|run_structured_cells|, only differing in the type of kernel function. The argument \lstinline|edge_kernel| is a kernel function which takes as arguments:
\begin{itemize}
\item the ids of the two neighbouring vertices
\item the ids of the two neighbouring cells
\item the ids of the vertex neighbours of the aforementioned cells
\item The forwarded arguments, as described above.
\end{itemize}


\section{Airfoil computation}
\texttt{run-airfoil-computation.cc} is a rewrite of the airfoil computation found in OP2~\cite{op2airfoil}. The computation is described in more detail in subsection~\ref{subsec:airfoil-computation}.


\section{Chapter summary}
In this chapter we detailed important aspects of our prototype implementation, including the mesh storage format and third-party libraries used. We described the important function that have been implemented, and crucially the scope of our implementation within our general discussion of structure detection.