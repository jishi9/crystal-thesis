% Structured nodes matrix
\newcommand{\Structured}{V_{structured}}

\newcommand{\VertexSet}{V}
\newcommand{\CellSet}{C}

\newcommand{\Visited}{V_{visited}}

% Adjacency relation
\newcommand{\AdjVV}{Adj_{\VertexSet\VertexSet}}

% Adjacency Vertices to cells
\newcommand{\AdjVC}{Adj_{\VertexSet\CellSet}}


% Cardinality of
\newcommand{\card}[1]{\left\vert{#1}\right\vert}

% powerset of
\newcommand{\powerset}[1]{\mathcal P \left({#1}\right)}


% nameref with lowercase
\newcommand{\lnameref}[1]{%
\bgroup
\let\nmu\MakeLowercase
\nameref{#1}\egroup}

% Superscript ref (for footnotes for e.g.)
\newcommand{\footref}[1]{\textsuperscript{\ref{#1}}}


% Tikz function: compute absolute position
% row, delta-sy, y-offset
\newcommand{\rctoxy}[3][3=0]{#1*#2 + #3}


% Image width for captions
\newcommand{\imagewidth}{0.8\textwidth}


% Z character
\newlength\theight
\settoheight{\theight}{t}

\newsavebox{\zee}
\begin{lrbox}{\zee}
\begin{tikzpicture}%
\draw (0,0) -- (\theight,0) -- (0, -\theight) -- (\theight, -\theight);%
\end{tikzpicture}%
\end{lrbox}

% Line that can go anywhere, structured region or outside
\tikzset{%
    anywhere/.style={
		decorate,
		decoration={
		    snake,
		    segment length=4,
		    amplitude=.9,post=lineto,
		    post length=2pt
		}
	}
}

% Line that can only go outside the structured region
\tikzset{outside/.style={dotted}}


% Ellipsis
\tikzset{ellipsis/.style={loosely dotted}}

% Structured
\tikzset{structured/.style={}}

% Category data for plots
% http://tex.stackexchange.com/questions/63335/pgfplots-using-strings-from-data-table-as-x-axis-labels-in-bar-chart
\makeatletter
\pgfplotsset{
    /pgfplots/flexible yticklabels from table/.code n args={3}{%
        \pgfplotstableread[#3]{#1}\coordinate@table
        \pgfplotstablegetcolumn{#2}\of{\coordinate@table}\to\pgfplots@yticklabels
        \let\pgfplots@yticklabel=\pgfplots@user@ticklabel@list@y
    }
}
\makeatother


%% Node labelling functions

% row, col, rowoffset, coloffset
\newcommand{\plainlabelnode}[4]{
\pgfmathtruncatemacro{\rowlabel}{#1+#3}\pgfmathtruncatemacro{\collabel}{#2+#4}$n_{\rowlabel,\collabel}$}

% row, col, rowoffset, coloffset, column-variable
\newcommand{\varcollabelnode}[5]{
\pgfmathtruncatemacro{\row}{#1+#3}\pgfmathtruncatemacro{\col}{#2+#4}\pgfmathtruncatemacro{\colzero}{\col-1}\newcommand{\collabel}{\ifnumcomp{\colzero}{=}{0}{#5}{\ifnumcomp{\colzero}{>}{0}{#5+\colzero}{#5\colzero}}}$n_{\row,\collabel}$}

% row, col, rowoffset, coloffset, row-variable
\newcommand{\varrowlabelnode}[5]{\pgfmathtruncatemacro{\row}{#1+#3}\pgfmathtruncatemacro{\rowzero}{\row-1}\pgfmathtruncatemacro{\col}{#2+#4}\newcommand{\rowlabel}{\ifnumcomp{\rowzero}{=}{0}{#5}{\ifnumcomp{\rowzero}{>}{0}{#5+\rowzero}{#5\rowzero}}}$n_{\rowlabel,\col}$}


% row, col, rowoffset, coloffset, row-variable, col-variable
\newcommand{\varlabelnode}[6]{\pgfmathtruncatemacro{\row}{#1+#3}\pgfmathtruncatemacro{\rowzero}{\row-1}\pgfmathtruncatemacro{\col}{#2+#4}\pgfmathtruncatemacro{\colzero}{\col-1}\newcommand{\rowlabel}{\ifnumcomp{\rowzero}{=}{0}{#5}{\ifnumcomp{\rowzero}{>}{0}{#5+\rowzero}{#5\rowzero}}}\newcommand{\collabel}{\ifnumcomp{\colzero}{=}{0}{#6}{\ifnumcomp{\colzero}{>}{0}{#6+\colzero}{#6\colzero}}}$n_{\rowlabel,\collabel}$}



\pgfkeys{/tikz/.cd,% to set the path
  rows/.get=\krows,
  rows/.store in=\krows,
  cols/.get=\kcols,
  cols/.store in=\kcols,
  rowoffset/.initial=0,
  rowoffset/.get=\krowoffset,
  rowoffset/.store in=\krowoffset,
  coloffset/.initial=0,
  coloffset/.get=\kcoloffset,
  coloffset/.store in=\kcoloffset,
  labeler/.get=\klabeler,
  labeler/.store in=\klabeler,
  labelerA/.get=\klabelerA,
  labelerA/.store in=\klabelerA,
  labelerB/.get=\klabelerB,
  labelerB/.store in=\klabelerB,
  labelerC/.get=\klabelerC,
  labelerC/.store in=\klabelerC,
  labelerD/.get=\klabelerD,
  labelerD/.store in=\klabelerD,
  northborder/.initial=outside,
  northborder/.get=\knorthborder,
  northborder/.store in=\knorthborder,
  southborder/.initial=outside,
  southborder/.get=\ksouthborder,
  southborder/.store in=\ksouthborder,
  eastborder/.initial=outside,
  eastborder/.get=\keastborder,
  eastborder/.store in=\keastborder,
  westborder/.initial=outside,
  westborder/.get=\kwestborder,
  westborder/.store in=\kwestborder,
}

%% Draws a grid - num rows, num cols, row offset, col offset
\newcommand{\drawgrid}[1]{{
     \tikzset{#1}


	\newcommand{\maxrows}{\krows}
	\newcommand{\maxcols}{\kcols}
	\newcommand{\rowoffset}{\krowoffset}
	\newcommand{\coloffset}{\kcoloffset}
	% Argument is a function: r,c -> node label
	\newcommand{\labeler}{\klabeler}

	\foreach \row in {1,...,\maxrows} {
		\pgfmathsetmacro{\ypos}{-2 * \row + \rowoffset}
		\pgfmathtruncatemacro{\prevrow}{\row - 1}

		\foreach \col in {1,...,\maxcols} {
			\pgfmathsetmacro{\xpos}{2 * \col + \coloffset}
			\pgfmathtruncatemacro{\prevcol}{\col - 1}
			\newcommand{\thisnode}{(n \row \space \col)}

			% Get node label
			\ifstrempty{\klabelerA}{
				\newcommand{\nodelabel}{\labeler{\row}{\col}}
			}
			\ifstrempty{\klabelerB} {
				\newcommand{\nodelabel}{\labeler{\row}{\col}{\klabelerA}}
			}
			\ifstrempty{\klabelerC} {
				\newcommand{\nodelabel}{\labeler{\row}{\col}{\klabelerA}{\klabelerB}}
			}
			\ifstrempty{\klabelerD} {
				\newcommand{\nodelabel}{\labeler{\row}{\col}{\klabelerA}{\klabelerB}{\klabelerC}}
			}
			{
				\newcommand{\nodelabel}{\labeler{\row}{\col}{\klabelerA}{\klabelerB}{\klabelerC}{\klabelerD}}
			}
			% Create node
			\node (n \row \space \col) at (\xpos,\ypos) {\nodelabel};

			% Line from node to the previous horizontal node
			\ifnumcomp{\col}{>}{1} {
				\draw (n \row \space \prevcol) -- \thisnode;
			}

			% Line from node to the previous vertical node
			\ifnumcomp{\row}{>}{1} {
				\draw (n \prevrow \space \col) -- \thisnode;
			}


			% West border lines
			\ifnumcomp{\col}{=}{1} {
				\draw[\kwestborder] \thisnode -- (\xpos-1, \ypos);
			}
			% East border lines
			\ifnumcomp{\col}{=}{\maxcols} {
				\draw[\keastborder] (\xpos+1, \ypos) -- \thisnode;
			}

			% North border lines
			\ifnumcomp{\row}{=}{1} {
				\draw[\knorthborder] \thisnode -- (\xpos, \ypos+1);
			}
			% South border lines
			\ifnumcomp{\row}{=}{\maxrows} {
				\draw[\ksouthborder] (\xpos, \ypos-1) -- \thisnode;
			}
		}
	}
}}
\pgfkeys{/tikz/.cd,% to set the path
  num/.get=\knum,
  num/.store in=\knum,
  rowoffset/.initial=0,
  rowoffset/.get=\krowoffset,
  rowoffset/.store in=\krowoffset,
  coloffset/.initial=0,
  coloffset/.get=\kcoloffset,
  coloffset/.store in=\kcoloffset,
}


%% Draws a row of vertical ellipses - num cols, row offset, col offset
\newcommand{\drawellipsisrow}[1]{{
	\tikzset{#1}

	\newcommand{\numellipses}{\knum}
	\newcommand{\rowoffset}{\krowoffset}
	\newcommand{\coloffset}{\kcoloffset}

	\foreach \col in {1,...,\numellipses} {
		\pgfmathsetmacro{\xpos}{2 * \col + \coloffset}
		\pgfmathsetmacro{\ypos}{\rowoffset}

		\draw[ellipsis] (\xpos, \ypos) -- (\xpos, \ypos-1);
	}
}}



\pgfkeys{/tikz/.cd,% to set the path
  num/.get=\knum,
  num/.store in=\knum,
  rowoffset/.initial=0,
  rowoffset/.get=\krowoffset,
  rowoffset/.store in=\krowoffset,
  coloffset/.initial=0,
  coloffset/.get=\kcoloffset,
  coloffset/.store in=\kcoloffset,
}


%% Draws a column of horizontal ellipses - num rows, row offset, col offset
\newcommand{\drawellipsiscol}[1]{{
	\tikzset{#1}

	\newcommand{\numellipses}{\knum}
	\newcommand{\rowoffset}{\krowoffset}
	\newcommand{\coloffset}{\kcoloffset}

	\foreach \row in {1,...,\numellipses} {
		\pgfmathsetmacro{\xpos}{\coloffset}
		\pgfmathsetmacro{\ypos}{-2 * \row + \rowoffset}

		\draw[ellipsis] (\xpos, \ypos) -- (\xpos+1, \ypos);
	}
}}
% Based on http://tex.stackexchange.com/questions/123719/drawing-a-large-binary-matrix-as-colored-grid-in-tikz

\definecolor{neighbourstructurecolor}{RGB}{174,190,226}
\makeatletter
\tikzset{
    blank color/.initial=white,
    blank color/.get=\zerocol,
    blank color/.store in=\zerocol,
    detected color/.initial=neighbourstructurecolor,
    detected color/.get=\onecol,
    detected color/.store in=\onecol,
    blocked color/.initial=black,
    blocked color/.get=\twocol,
    blocked color/.store in=\twocol,
    alt color/.initial=blue,
    alt color/.get=\threecol,
    alt color/.store in=\threecol,
    cell wd/.initial=1ex,
    cell wd/.get=\cellwd,
    cell wd/.store in=\cellwd,
    cell ht/.initial=1ex,
    cell ht/.get=\cellht,
    cell ht/.store in=\cellht,
}

\newcommand{\drawmatrix}[2][]{
\medskip
\begin{tikzpicture}[#1]
  \pgfplotstableforeachcolumn#2\as\col{
    \pgfplotstableforeachcolumnelement{\col}\of#2\as\colcnt{%
      \ifnum\colcnt=0
        \draw[draw=gray,very thin,fill=\zerocol]($ -\pgfplotstablerow*(0,\cellht) + \col*(\cellwd,0) $) rectangle+(\cellwd,\cellht);
      \fi
      \ifnum\colcnt=1
        \draw[draw=gray,very thin,fill=\onecol]($ -\pgfplotstablerow*(0,\cellht) + \col*(\cellwd,0) $) rectangle+(\cellwd,\cellht);
      \fi
      \ifnum\colcnt=2
        \fill[pattern=dots]($ -\pgfplotstablerow*(0,\cellht) + \col*(\cellwd,0) $) rectangle+(\cellwd,\cellht);
      \fi
      \ifnum\colcnt=3
        \draw[draw=gray,very thin,fill=\threecol]($ -\pgfplotstablerow*(0,\cellht) + \col*(\cellwd,0) $) rectangle+(\cellwd,\cellht);
      \fi
      \ifnum\colcnt=5
        \draw[draw=gray,very thin,fill=\onecol]($ -\pgfplotstablerow*(0,\cellht) + \col*(\cellwd,0) $) rectangle+(\cellwd,\cellht);
        \node[rectangle, minimum width=\cellwd, minimum height=\cellht] at ($ -\pgfplotstablerow*(0,\cellht) + \col*(\cellwd,0) + (\cellwd/2,\cellht/2) $) {A};
      \fi
      \ifnum\colcnt=6
        \draw[draw=gray,very thin,fill=\onecol]($ -\pgfplotstablerow*(0,\cellht) + \col*(\cellwd,0) $) rectangle+(\cellwd,\cellht);
        \node[rectangle, minimum width=\cellwd, minimum height=\cellht] at ($ -\pgfplotstablerow*(0,\cellht) + \col*(\cellwd,0) + (\cellwd/2,\cellht/2) $) {B};
      \fi
      \ifnum\colcnt=7
        \draw[draw=gray,very thin,fill=\onecol]($ -\pgfplotstablerow*(0,\cellht) + \col*(\cellwd,0) $) rectangle+(\cellwd,\cellht);
        \node[rectangle, minimum width=\cellwd, minimum height=\cellht] at ($ -\pgfplotstablerow*(0,\cellht) + \col*(\cellwd,0) + (\cellwd/2,\cellht/2) $) {C};
      \fi
      \ifnum\colcnt=9
        \draw[draw=gray,very thin,fill=\zerocol]($ -\pgfplotstablerow*(0,\cellht) + \col*(\cellwd,0) $) rectangle+(\cellwd,\cellht);
        \node[rectangle, text=red] at ($ -\pgfplotstablerow*(0,\cellht) + \col*(\cellwd,0) + (\cellwd/2,\cellht/2) $) {$\mathlarger{\mathlarger{\mathlarger{\mathlarger\times}}}$};
      \fi
    }
  }
\end{tikzpicture}
\medskip
}
\makeatother

\newcommand{\drawpermutes}{

	\edef\lecounter{0}
	\edef\xdelta{1}
	\edef\ydelta{1}

	% Permutations generated by supp/arrow-permutation.py
	\foreach \w/\x/\y/\z in {a/c/b/d, a/c/d/b, a/d/b/c, a/b/d/c, a/d/c/b, a/b/c/d, b/d/a/c, b/d/c/a, b/c/a/d, b/a/c/d, b/c/d/a, b/a/d/c, c/a/d/b, c/a/b/d, c/b/d/a, c/d/b/a, c/b/a/d, c/d/a/b, d/b/c/a, d/b/a/c, d/a/c/b, d/c/a/b, d/a/b/c, d/c/b/a}{

		\pgfmathtruncatemacro\xpos{2 * int(mod(\lecounter, 6))}
		\pgfmathtruncatemacro\ypos{2 * int(\lecounter / 6)}

		% Points
		\coordinate (a) at (\xpos, \ypos);
		\coordinate (b) at (\xpos+\xdelta, \ypos);
		\coordinate (d) at (\xpos, \ypos+\ydelta);
		\coordinate (c) at (\xpos+\xdelta, \ypos+\ydelta);

		% Draw lines
		\draw (\w) -- (\x) -- (\y);
		\draw[-{Stealth[length=3mm]}]  (\y) -- (\z);

		% \lecounter++
		\pgfmathtruncatemacro\newcounter{\lecounter + 1}
		\xdef\lecounter{\newcounter}
	}
}




% Side-by-side caption
\newcommand{\sidebyside}[2]{
	\begin{subfigure}[b]{0.45\textwidth}
	\centering
	#1
	\end{subfigure}
	\begin{subfigure}[b]{0.45\textwidth}
	\centering
	#2
	\end{subfigure}
}

\newcommand{\sidebysidevcenter}[2]{
	\begin{subfigure}[c]{0.45\textwidth}
	\centering
	#1
	\end{subfigure}
	\begin{subfigure}[c]{0.45\textwidth}
	\centering
	#2
	\end{subfigure}
}

\newcommand{\sidebysidecaptions}[2]{
	\begin{subfigure}[t]{0.45\textwidth}
	\centering
	#1
	\end{subfigure}\hfill
	\begin{subfigure}[t]{0.45\textwidth}
	\centering
	#2
	\end{subfigure}
}

\newcommand{\sidebysidevertical}[2]{
	\begin{subfigure}[b]{\textwidth}
	\centering
	#1
	\end{subfigure}
	\begin{subfigure}[b]{\textwidth}
	\centering
	#2
	\end{subfigure}
}

\newcommand{\sidebysideverticalnoncenter}[2]{
	\begin{subfigure}[b]{\textwidth}
	#1
	\end{subfigure}
	\begin{subfigure}[b]{\textwidth}
	#2
	\end{subfigure}
}

\newcommand{\sidebysidethree}[3]{
	\begin{subfigure}[b]{0.32\textwidth}
	\centering
	#1
	\end{subfigure}
	\begin{subfigure}[b]{0.32\textwidth}
	\centering
	#2
	\end{subfigure}
	\begin{subfigure}[b]{0.32\textwidth}
	\centering
	#3
	\end{subfigure}
}

\newcommand{\sidebysidethreevertical}[3]{
	\begin{subfigure}[b]{\textwidth}
	\centering
	#1
	\end{subfigure}
	\begin{subfigure}[b]{\textwidth}
	\centering
	#2
	\end{subfigure}
	\begin{subfigure}[b]{\textwidth}
	\centering
	#3
	\end{subfigure}
}

\newcommand{\sidebysidethreeverticaltop}[3]{
	\begin{subfigure}[t]{\textwidth}
	\centering
	#1
	\end{subfigure}
	\begin{subfigure}[t]{\textwidth}
	\centering
	#2
	\end{subfigure}
	\begin{subfigure}[t]{\textwidth}
	\centering
	#3
	\end{subfigure}
}

\newcommand{\sidebysidethreeverticalnoncenter}[3]{
	\begin{subfigure}[b]{\textwidth}
	#1
	\end{subfigure}
	\begin{subfigure}[b]{\textwidth}
	#2
	\end{subfigure}
	\begin{subfigure}[b]{\textwidth}
	#3
	\end{subfigure}
}

\newcommand{\sidebysidefour}[4]{
	\begin{subfigure}[b]{0.45\textwidth}
	\centering
	#1
	\end{subfigure}
	\begin{subfigure}[b]{0.45\textwidth}
	\centering
	#2
	\end{subfigure}
	\begin{subfigure}[b]{0.45\textwidth}
	\centering
	#3
	\end{subfigure}
	\begin{subfigure}[b]{0.45\textwidth}
	\centering
	#4
	\end{subfigure}
}


% Listings
\definecolor{codegreen}{rgb}{0,0.6,0}
\definecolor{codegray}{rgb}{0.5,0.5,0.5}
\definecolor{codepurple}{rgb}{0.58,0,0.82}
\definecolor{backcolour}{rgb}{0.95,0.95,0.92}

\lstdefinestyle{mystyle}{
    backgroundcolor=\color{backcolour},
    commentstyle=\color{codegreen},
    keywordstyle=\color{magenta},
    numberstyle=\tiny\color{codegray},
    stringstyle=\color{codepurple},
    basicstyle=\footnotesize,
    breakatwhitespace=false,
    breaklines=true,
    captionpos=b,
    keepspaces=true,
    numbers=left,
    numbersep=5pt,
    showspaces=false,
    showstringspaces=false,
    showtabs=false,
    tabsize=2
}

\lstset{style=mystyle}